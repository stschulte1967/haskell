% !TEX TS-program = pdflatex
% !TEX encoding = UTF-8 Unicode

% This is a simple template for a LaTeX document using the "article" class.
% See "book", "report", "letter" for other types of document.

\documentclass[11pt]{article} % use larger type; default would be 10pt

\usepackage[utf8]{inputenc} % set input encoding (not needed with XeLaTeX)

%%% Examples of Article customizations
% These packages are optional, depending whether you want the features they provide.
% See the LaTeX Companion or other references for full information.

%%% PAGE DIMENSIONS
\usepackage{geometry} % to change the page dimensions
\geometry{a4paper} % or letterpaper (US) or a5paper or....
% \geometry{margin=2in} % for example, change the margins to 2 inches all round
% \geometry{landscape} % set up the page for landscape
%   read geometry.pdf for detailed page layout information

\usepackage{graphicx} % support the \includegraphics command and options

% \usepackage[parfill]{parskip} % Activate to begin paragraphs with an empty line rather than an indent

%%% PACKAGES
\usepackage{booktabs} % for much better looking tables
\usepackage{array} % for better arrays (eg matrices) in maths
\usepackage{paralist} % very flexible & customisable lists (eg. enumerate/itemize, etc.)
\usepackage{verbatim} % adds environment for commenting out blocks of text & for better verbatim
\usepackage{subfig} % make it possible to include more than one captioned figure/table in a single float
% These packages are all incorporated in the memoir class to one degree or another...

%%% HEADERS & FOOTERS
\usepackage{fancyhdr} % This should be set AFTER setting up the page geometry
\pagestyle{fancy} % options: empty , plain , fancy
\renewcommand{\headrulewidth}{0pt} % customise the layout...
\lhead{}\chead{}\rhead{}
\lfoot{}\cfoot{\thepage}\rfoot{}

%%% SECTION TITLE APPEARANCE
\usepackage{sectsty}
\allsectionsfont{\sffamily\mdseries\upshape} % (See the fntguide.pdf for font help)
% (This matches ConTeXt defaults)

%%% ToC (table of contents) APPEARANCE
\usepackage[nottoc,notlof,notlot]{tocbibind} % Put the bibliography in the ToC
\usepackage[titles,subfigure]{tocloft} % Alter the style of the Table of Contents
\renewcommand{\cftsecfont}{\rmfamily\mdseries\upshape}
\renewcommand{\cftsecpagefont}{\rmfamily\mdseries\upshape} % No bold!

\usepackage{listings}
\usepackage{amsthm}

\theoremstyle{definition}
\newtheorem{axiom}{Definition}
%%% END Article customizations

%%% The "real" document content comes below...

\title{The Little Typer}
\author{Stephan Schulte}
%\date{} % Activate to display a given date or no date (if empty),
         % otherwise the current date is printed 
\setlength{\parindent}{0pt}
\begin{document}
\maketitle

\section{The More Things Change, the More They Stay the Same}

\begin{axiom}[The Law of Tick Marks]
A tick mark directly followed by one or more letters and hyphens is an \textit{Atom}.
\end{axiom}

\begin{axiom}[The Commandment of Tick Marks]
Two expressions are the same \textit{Atom} if their values are tick marks followed by identical letters and hyphens.
\end{axiom}

\begin{axiom}[The Law of \textit{Atom} ]
\textit{Atom} is a type.
\end{axiom}

\begin{axiom}[The Four Forms of Judgement]\hfill \\
\begin{tabular} {ll}
1. $\ldots$ is a $\ldots$. & 2. $\ldots$ is the same $\ldots$ as $\ldots$.\\
3. $\ldots$ is a type. & 4. $\ldots$ and $\ldots$ are the same type.
\end{tabular}
\end{axiom}

\begin{axiom}[Normal Forms]
Given a type, every expression described by that type has a \textit{normal form}, which is the most direct way of writing it. If two expressions are the same, then they have identical normal forms, and if they have identical normal forms, then they are the same. 
\end{axiom}

\begin{axiom}[Normal Forms and Types]
Sameness is always according to a type, so normal forms are also determined by a type.
\end{axiom}

\begin{axiom}[The first commandment of \textit{cons}]
Two \textit{cons}-expressions are the same (Pair A D) if their cars are the same A and their cdrs are the same D. Here, A and D stand for any type.
\end{axiom}

\begin{axiom}[Normal Forms of Types]
Every expression that is a type has a normal form., which is the most direct way of writing that type. If two expressions are the same type, then they have identical normal forms, and if two types have identical normal forms, then they are the same type.
\end{axiom}

\begin{axiom}[Claims before Definitions]
Using define to associate a name with an expression required that the expression's type has previously been associated with the name using claim.
\end{axiom}

\begin{axiom}[Values]
An expression with a constructor at the top is called a \textit{value}.
\end{axiom}

\begin{axiom}[Values and Normal Forms]
Not every value is in normal form. This is because the arguments to a constructor need not be normal. Each expression has only one normal form, but it is sometimes possible to write it as a value in more than one way.
\end{axiom}

\begin{axiom}[Everything is an Expression]
In Pie, values are also expressions. Evaluation in Pie finds an expression, not some other kind of thing.
\end{axiom}

\begin{axiom}[The commandment of \textit{zero}]
\textit{zero} is the same Nat as \textit{zero}. 
\end{axiom}

\begin{axiom}[The commandment of \textit{add1}]
If $n$ is the same Nat as $k$, then (add1 n) is the same Nat as (add1 k).
\end{axiom}

\begin{axiom}[Definitions are Foreever]
Once a name has been claimed, it cannot be reclaimed, and once a name has been defined it cannot be redefined.
\end{axiom}

\section{Doin' What Comes Naturally}

\begin{axiom}[Constructors and Eliminators]
Constructors build values, and eliminators take apart values build by constructors.
\end{axiom}

\begin{axiom}[Eliminating Functions]
Applying a function to arguments is the eliminator for functions.
\end{axiom}

\begin{axiom}[The Initial Law of Application]
If $f$ is an $()\rightarrow Y X)$ and $arg$ is a $Y$, then $(f arg)$ is a $X$.
\end{axiom}

\begin{axiom}[The Initial First Commandment of $\lambda$]
Two $\lambda$-expressions that exprect the same number of arguments are the same if their bodies are the same after consistently renaming their variables.
\end{axiom}

\begin{axiom}[The Initial Second Commandment of $\lambda$]
If $f$ is an $(\rightarrow Y X)$, then $f$ is the same $(\rightarrow Y X)$ as $(\lambda (y) (f(y))$, as long as $y$ does not occur un $f$.  
\end{axiom}

\begin{axiom}[The Law of Renaming Variables]
Consistently renaming variables can't change the meaning of anything.
\end{axiom}

\begin{axiom}[The Commandment of Neutral Expressions]
Neutral expressions that are written identically are the same, no matter their type.
\end{axiom}

\begin{axiom}[The Law and Commandment of define]
Following $(claim\, name\, X)$ and $(define\, name\, expr)$, if expr is an X, than name is an X and name is the same X as expr.
\end{axiom}

\begin{axiom}[The Second Commandment of cons]
If $p$ is a $(Pair\, A\, D)$, then it is the same $(Pair\, A\, D)$ as $(cons (car\, p) (cdr\, p))$.
\end{axiom}

\begin{axiom}[Names in Definitions]
In Pie, only names that are not already used, whether for constructors, eliminators, or previous definitions, can be used with claim or define.
\end{axiom}

\begin{axiom}[Dim Names]
Unused names are written dimly, but they do need to be there.
\end{axiom}

\begin{axiom}[The Law if which-Nat]
If target is a Nat, base is an X, and step is an $(\rightarrow Nat\, X)$, then
\begin{lstlisting}
(which-Nat target
	base
	step)
\end{lstlisting}
is an X.
\end{axiom}

\begin{axiom}[The First Commandment of which-Nat]
If (which-Nat zero base step) in an X, than it is the same X as base.
\end{axiom}

\begin{axiom}[The Second Commandment of which-Nat]
If (which-Nat (add1 n) base step) in an X, than it is the same X as (step n).
\end{axiom}

\begin{axiom}[Type Values]
An expression that is described by a type is a vlaue when it has a constructor at its top. Similarly, an espression that is a type is a value when it has a type constructor at its top.
\end{axiom}

\begin{axiom}[Every U is a Type]
Every expression described by U is a type, but not every type is described by U.
\end{axiom}

\begin{axiom}[Definitions are Unnecessary]
Everything can be done without definitions, but they do improve understanding.
\end{axiom}

\end{document}
